\documentclass[11pt]{article}
\usepackage[utf8]{inputenc}
\usepackage{moreverb}
\usepackage{mathtools}
\usepackage{amsfonts}
\usepackage{pifont}
\usepackage{ngerman}
\newcommand{\tick}{\ding{52}}
\newcommand{\Nat}{\mathbb{N}}
\newcommand{\Real}{\mathbb{R}}
\newcommand{\pset}{\mathcal{P}}
\renewcommand{\implies}{\Rightarrow}
\renewcommand{\iff}{\Leftrightarrow}
\title{\textbf{Protokoll SysPrak\\Gruppe 23}}
\author{}
\date{\today}
\begin{document}

\maketitle

\section*{Stil}
Beschluss, als Code-Stil einen modifizierten Java-Stil zu nutzen mit Eigenschaften, welche sich aus dem folgendem Code-Beispiel ergeben:
\begin{verbatim}
inf main() {
 int i = 10;
 int longNameVariable;
 
 if(i == 1) {
   // do things
  } else {
   // do other things
  }
  
  for(int j = 0; j < i; j++) {
   // do loopy things
  }
}
\end{verbatim}
Kommentare können in Deutsch, Englisch und Denglisch verfasst werden.\\
Kommentar-Stil: Jeder Methode wird ein Mehrzeilenkommentar vorangestellt im doxygen-Stil mit Informationen zu Input, Handling und Output.
Variablen- und Funktionsnamen sind in Englisch zu halten.\\
\section*{Config und Versioning}
Github-Versionierung des Projekts im Repo PPP12345/Schwulenverein-SanFrancisco-Gruppe23 mit forking und pull requests in klassischem Projekt-Stil.\\
Genutzter Standard: C99.\\
\section*{Good Practices}
Nutzung von EXIT\_FAILURE und EXIT\_SUCCESS aus der stdlib.h.\\
Keine Nutzung der stdbool.h, stattdessen int type.\\
Hauptkommunikationsmittel ist die Mailingliste.\\
Es wird nichts ins Repo repusht (u.U. mit merge request), was nicht lauffähig ist.\\
FIRST PULL, THEN PUSH!\\
Git commits sollen zusammenfassende Kommentare enthalten. Bei größeren Kommentaren soll eine Erklärung über die Mailingliste erfolgen.
\end{document}
